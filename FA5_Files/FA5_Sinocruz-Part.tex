% Options for packages loaded elsewhere
\PassOptionsToPackage{unicode}{hyperref}
\PassOptionsToPackage{hyphens}{url}
%
\documentclass[
]{article}
\usepackage{amsmath,amssymb}
\usepackage{iftex}
\ifPDFTeX
  \usepackage[T1]{fontenc}
  \usepackage[utf8]{inputenc}
  \usepackage{textcomp} % provide euro and other symbols
\else % if luatex or xetex
  \usepackage{unicode-math} % this also loads fontspec
  \defaultfontfeatures{Scale=MatchLowercase}
  \defaultfontfeatures[\rmfamily]{Ligatures=TeX,Scale=1}
\fi
\usepackage{lmodern}
\ifPDFTeX\else
  % xetex/luatex font selection
\fi
% Use upquote if available, for straight quotes in verbatim environments
\IfFileExists{upquote.sty}{\usepackage{upquote}}{}
\IfFileExists{microtype.sty}{% use microtype if available
  \usepackage[]{microtype}
  \UseMicrotypeSet[protrusion]{basicmath} % disable protrusion for tt fonts
}{}
\makeatletter
\@ifundefined{KOMAClassName}{% if non-KOMA class
  \IfFileExists{parskip.sty}{%
    \usepackage{parskip}
  }{% else
    \setlength{\parindent}{0pt}
    \setlength{\parskip}{6pt plus 2pt minus 1pt}}
}{% if KOMA class
  \KOMAoptions{parskip=half}}
\makeatother
\usepackage{xcolor}
\usepackage[margin=1in]{geometry}
\usepackage{color}
\usepackage{fancyvrb}
\newcommand{\VerbBar}{|}
\newcommand{\VERB}{\Verb[commandchars=\\\{\}]}
\DefineVerbatimEnvironment{Highlighting}{Verbatim}{commandchars=\\\{\}}
% Add ',fontsize=\small' for more characters per line
\usepackage{framed}
\definecolor{shadecolor}{RGB}{248,248,248}
\newenvironment{Shaded}{\begin{snugshade}}{\end{snugshade}}
\newcommand{\AlertTok}[1]{\textcolor[rgb]{0.94,0.16,0.16}{#1}}
\newcommand{\AnnotationTok}[1]{\textcolor[rgb]{0.56,0.35,0.01}{\textbf{\textit{#1}}}}
\newcommand{\AttributeTok}[1]{\textcolor[rgb]{0.13,0.29,0.53}{#1}}
\newcommand{\BaseNTok}[1]{\textcolor[rgb]{0.00,0.00,0.81}{#1}}
\newcommand{\BuiltInTok}[1]{#1}
\newcommand{\CharTok}[1]{\textcolor[rgb]{0.31,0.60,0.02}{#1}}
\newcommand{\CommentTok}[1]{\textcolor[rgb]{0.56,0.35,0.01}{\textit{#1}}}
\newcommand{\CommentVarTok}[1]{\textcolor[rgb]{0.56,0.35,0.01}{\textbf{\textit{#1}}}}
\newcommand{\ConstantTok}[1]{\textcolor[rgb]{0.56,0.35,0.01}{#1}}
\newcommand{\ControlFlowTok}[1]{\textcolor[rgb]{0.13,0.29,0.53}{\textbf{#1}}}
\newcommand{\DataTypeTok}[1]{\textcolor[rgb]{0.13,0.29,0.53}{#1}}
\newcommand{\DecValTok}[1]{\textcolor[rgb]{0.00,0.00,0.81}{#1}}
\newcommand{\DocumentationTok}[1]{\textcolor[rgb]{0.56,0.35,0.01}{\textbf{\textit{#1}}}}
\newcommand{\ErrorTok}[1]{\textcolor[rgb]{0.64,0.00,0.00}{\textbf{#1}}}
\newcommand{\ExtensionTok}[1]{#1}
\newcommand{\FloatTok}[1]{\textcolor[rgb]{0.00,0.00,0.81}{#1}}
\newcommand{\FunctionTok}[1]{\textcolor[rgb]{0.13,0.29,0.53}{\textbf{#1}}}
\newcommand{\ImportTok}[1]{#1}
\newcommand{\InformationTok}[1]{\textcolor[rgb]{0.56,0.35,0.01}{\textbf{\textit{#1}}}}
\newcommand{\KeywordTok}[1]{\textcolor[rgb]{0.13,0.29,0.53}{\textbf{#1}}}
\newcommand{\NormalTok}[1]{#1}
\newcommand{\OperatorTok}[1]{\textcolor[rgb]{0.81,0.36,0.00}{\textbf{#1}}}
\newcommand{\OtherTok}[1]{\textcolor[rgb]{0.56,0.35,0.01}{#1}}
\newcommand{\PreprocessorTok}[1]{\textcolor[rgb]{0.56,0.35,0.01}{\textit{#1}}}
\newcommand{\RegionMarkerTok}[1]{#1}
\newcommand{\SpecialCharTok}[1]{\textcolor[rgb]{0.81,0.36,0.00}{\textbf{#1}}}
\newcommand{\SpecialStringTok}[1]{\textcolor[rgb]{0.31,0.60,0.02}{#1}}
\newcommand{\StringTok}[1]{\textcolor[rgb]{0.31,0.60,0.02}{#1}}
\newcommand{\VariableTok}[1]{\textcolor[rgb]{0.00,0.00,0.00}{#1}}
\newcommand{\VerbatimStringTok}[1]{\textcolor[rgb]{0.31,0.60,0.02}{#1}}
\newcommand{\WarningTok}[1]{\textcolor[rgb]{0.56,0.35,0.01}{\textbf{\textit{#1}}}}
\usepackage{graphicx}
\makeatletter
\def\maxwidth{\ifdim\Gin@nat@width>\linewidth\linewidth\else\Gin@nat@width\fi}
\def\maxheight{\ifdim\Gin@nat@height>\textheight\textheight\else\Gin@nat@height\fi}
\makeatother
% Scale images if necessary, so that they will not overflow the page
% margins by default, and it is still possible to overwrite the defaults
% using explicit options in \includegraphics[width, height, ...]{}
\setkeys{Gin}{width=\maxwidth,height=\maxheight,keepaspectratio}
% Set default figure placement to htbp
\makeatletter
\def\fps@figure{htbp}
\makeatother
\setlength{\emergencystretch}{3em} % prevent overfull lines
\providecommand{\tightlist}{%
  \setlength{\itemsep}{0pt}\setlength{\parskip}{0pt}}
\setcounter{secnumdepth}{-\maxdimen} % remove section numbering
\ifLuaTeX
  \usepackage{selnolig}  % disable illegal ligatures
\fi
\usepackage{bookmark}
\IfFileExists{xurl.sty}{\usepackage{xurl}}{} % add URL line breaks if available
\urlstyle{same}
\hypersetup{
  pdftitle={Formative Assessment 5},
  pdfauthor={SINOCRUZ, ARVIE},
  hidelinks,
  pdfcreator={LaTeX via pandoc}}

\title{Formative Assessment 5}
\author{SINOCRUZ, ARVIE}
\date{2025-10-05}

\begin{document}
\maketitle

library(knitr) library(tinytex)

\subsection{Problem 8.34}\label{problem-8.34}

\begin{Shaded}
\begin{Highlighting}[]
\CommentTok{\# Given values}
\NormalTok{n }\OtherTok{\textless{}{-}} \DecValTok{200}
\NormalTok{probab }\OtherTok{\textless{}{-}} \FloatTok{0.5}
\NormalTok{q }\OtherTok{\textless{}{-}} \DecValTok{1} \SpecialCharTok{{-}}\NormalTok{ probab}

\NormalTok{sd\_p }\OtherTok{\textless{}{-}} \FunctionTok{sqrt}\NormalTok{(probab }\SpecialCharTok{*}\NormalTok{ q }\SpecialCharTok{/}\NormalTok{ n)  }\CommentTok{\# Standard deviation of sampling distribution}

\NormalTok{probab\_a }\OtherTok{\textless{}{-}} \FunctionTok{pnorm}\NormalTok{(}\FloatTok{0.40}\NormalTok{, }\AttributeTok{mean =}\NormalTok{ probab, }\AttributeTok{sd =}\NormalTok{ sd\_p)}

\NormalTok{probab\_b }\OtherTok{\textless{}{-}} \FunctionTok{pnorm}\NormalTok{(}\FloatTok{0.57}\NormalTok{, }\AttributeTok{mean =}\NormalTok{ probab, }\AttributeTok{sd =}\NormalTok{ sd\_p) }\SpecialCharTok{{-}} \FunctionTok{pnorm}\NormalTok{(}\FloatTok{0.43}\NormalTok{, }\AttributeTok{mean =}\NormalTok{ probab, }\AttributeTok{sd =}\NormalTok{ sd\_p)}

\NormalTok{probab\_c }\OtherTok{\textless{}{-}} \DecValTok{1} \SpecialCharTok{{-}} \FunctionTok{pnorm}\NormalTok{(}\FloatTok{0.54}\NormalTok{, }\AttributeTok{mean =}\NormalTok{probab, }\AttributeTok{sd =}\NormalTok{ sd\_p)}

\CommentTok{\# Display results}
\FunctionTok{cat}\NormalTok{(}\StringTok{"Standard deviation (σₚ̂):"}\NormalTok{, }\FunctionTok{round}\NormalTok{(sd\_p, }\DecValTok{4}\NormalTok{), }\StringTok{"}\SpecialCharTok{\textbackslash{}n\textbackslash{}n}\StringTok{"}\NormalTok{)}
\end{Highlighting}
\end{Shaded}

\begin{verbatim}
## Standard deviation (σₚ̂): 0.0354
\end{verbatim}

\begin{Shaded}
\begin{Highlighting}[]
\FunctionTok{cat}\NormalTok{(}\StringTok{"(a) P(p̂ \textless{} 0.40) ="}\NormalTok{, }\FunctionTok{round}\NormalTok{(probab\_a, }\DecValTok{4}\NormalTok{), }\StringTok{"}\SpecialCharTok{\textbackslash{}n}\StringTok{"}\NormalTok{)}
\end{Highlighting}
\end{Shaded}

\begin{verbatim}
## (a) P(p̂ < 0.40) = 0.0023
\end{verbatim}

\begin{Shaded}
\begin{Highlighting}[]
\FunctionTok{cat}\NormalTok{(}\StringTok{"(b) P(0.43 \textless{} p̂ \textless{} 0.57) ="}\NormalTok{, }\FunctionTok{round}\NormalTok{(probab\_b, }\DecValTok{4}\NormalTok{), }\StringTok{"}\SpecialCharTok{\textbackslash{}n}\StringTok{"}\NormalTok{)}
\end{Highlighting}
\end{Shaded}

\begin{verbatim}
## (b) P(0.43 < p̂ < 0.57) = 0.9523
\end{verbatim}

\begin{Shaded}
\begin{Highlighting}[]
\FunctionTok{cat}\NormalTok{(}\StringTok{"(c) P(p̂ \textgreater{} 0.54) ="}\NormalTok{, }\FunctionTok{round}\NormalTok{(probab\_c, }\DecValTok{4}\NormalTok{), }\StringTok{"}\SpecialCharTok{\textbackslash{}n\textbackslash{}n}\StringTok{"}\NormalTok{)}
\end{Highlighting}
\end{Shaded}

\begin{verbatim}
## (c) P(p̂ > 0.54) = 0.1289
\end{verbatim}

\begin{Shaded}
\begin{Highlighting}[]
\NormalTok{x }\OtherTok{\textless{}{-}} \FunctionTok{seq}\NormalTok{(}\FloatTok{0.35}\NormalTok{, }\FloatTok{0.65}\NormalTok{, }\AttributeTok{by =} \FloatTok{0.001}\NormalTok{)}
\NormalTok{y }\OtherTok{\textless{}{-}} \FunctionTok{dnorm}\NormalTok{(x, }\AttributeTok{mean =}\NormalTok{ probab, }\AttributeTok{sd =}\NormalTok{ sd\_p)}

\FunctionTok{plot}\NormalTok{(x, y, }\AttributeTok{type =} \StringTok{"l"}\NormalTok{, }\AttributeTok{lwd =} \DecValTok{2}\NormalTok{, }\AttributeTok{col =} \StringTok{"blue"}\NormalTok{,}
     \AttributeTok{main =} \StringTok{"Sampling Distribution of Proportion (n = 200, probab = 0.5)"}\NormalTok{,}
     \AttributeTok{xlab =} \StringTok{"Sample Proportion (p̂)"}\NormalTok{,}
     \AttributeTok{ylab =} \StringTok{"Density"}\NormalTok{)}

\CommentTok{\# Vertical reference lines}
\FunctionTok{abline}\NormalTok{(}\AttributeTok{v =} \FunctionTok{c}\NormalTok{(}\FloatTok{0.40}\NormalTok{, }\FloatTok{0.43}\NormalTok{, }\FloatTok{0.54}\NormalTok{, }\FloatTok{0.57}\NormalTok{), }\AttributeTok{col =} \StringTok{"red"}\NormalTok{, }\AttributeTok{lty =} \DecValTok{2}\NormalTok{)}

\CommentTok{\# {-}{-}{-} (a) Shade area for p̂ \textless{} 0.40 {-}{-}{-}}
\NormalTok{x\_a }\OtherTok{\textless{}{-}} \FunctionTok{seq}\NormalTok{(}\FloatTok{0.35}\NormalTok{, }\FloatTok{0.40}\NormalTok{, }\AttributeTok{by =} \FloatTok{0.001}\NormalTok{)}
\NormalTok{y\_a }\OtherTok{\textless{}{-}} \FunctionTok{dnorm}\NormalTok{(x\_a, }\AttributeTok{mean =}\NormalTok{ probab, }\AttributeTok{sd =}\NormalTok{ sd\_p)}
\FunctionTok{polygon}\NormalTok{(}\FunctionTok{c}\NormalTok{(x\_a, }\FunctionTok{rev}\NormalTok{(x\_a)), }\FunctionTok{c}\NormalTok{(y\_a, }\FunctionTok{rep}\NormalTok{(}\DecValTok{0}\NormalTok{, }\FunctionTok{length}\NormalTok{(y\_a))), }\AttributeTok{col =} \StringTok{"lightcoral"}\NormalTok{, }\AttributeTok{border =} \ConstantTok{NA}\NormalTok{)}

\CommentTok{\# {-}{-}{-} (b) Shade area for 0.43 \textless{} p̂ \textless{} 0.57 {-}{-}{-}}
\NormalTok{x\_b }\OtherTok{\textless{}{-}} \FunctionTok{seq}\NormalTok{(}\FloatTok{0.43}\NormalTok{, }\FloatTok{0.57}\NormalTok{, }\AttributeTok{by =} \FloatTok{0.001}\NormalTok{)}
\NormalTok{y\_b }\OtherTok{\textless{}{-}} \FunctionTok{dnorm}\NormalTok{(x\_b, }\AttributeTok{mean =}\NormalTok{ probab, }\AttributeTok{sd =}\NormalTok{ sd\_p)}
\FunctionTok{polygon}\NormalTok{(}\FunctionTok{c}\NormalTok{(x\_b, }\FunctionTok{rev}\NormalTok{(x\_b)), }\FunctionTok{c}\NormalTok{(y\_b, }\FunctionTok{rep}\NormalTok{(}\DecValTok{0}\NormalTok{, }\FunctionTok{length}\NormalTok{(y\_b))), }\AttributeTok{col =} \StringTok{"lightgreen"}\NormalTok{, }\AttributeTok{border =} \ConstantTok{NA}\NormalTok{)}

\CommentTok{\# {-}{-}{-} (c) Shade area for p̂ \textgreater{} 0.54 {-}{-}{-}}
\NormalTok{x\_c }\OtherTok{\textless{}{-}} \FunctionTok{seq}\NormalTok{(}\FloatTok{0.54}\NormalTok{, }\FloatTok{0.65}\NormalTok{, }\AttributeTok{by =} \FloatTok{0.001}\NormalTok{)}
\NormalTok{y\_c }\OtherTok{\textless{}{-}} \FunctionTok{dnorm}\NormalTok{(x\_c, }\AttributeTok{mean =}\NormalTok{ probab, }\AttributeTok{sd =}\NormalTok{ sd\_p)}
\FunctionTok{polygon}\NormalTok{(}\FunctionTok{c}\NormalTok{(x\_c, }\FunctionTok{rev}\NormalTok{(x\_c)), }\FunctionTok{c}\NormalTok{(y\_c, }\FunctionTok{rep}\NormalTok{(}\DecValTok{0}\NormalTok{, }\FunctionTok{length}\NormalTok{(y\_c))), }\AttributeTok{col =} \StringTok{"lightblue"}\NormalTok{, }\AttributeTok{border =} \ConstantTok{NA}\NormalTok{)}

\CommentTok{\# Redraw main curve on top}
\FunctionTok{lines}\NormalTok{(x, y, }\AttributeTok{lwd =} \DecValTok{2}\NormalTok{, }\AttributeTok{col =} \StringTok{"black"}\NormalTok{)}

\CommentTok{\# Legend}
\FunctionTok{legend}\NormalTok{(}\StringTok{"topright"}\NormalTok{,}
       \AttributeTok{legend =} \FunctionTok{c}\NormalTok{(}\StringTok{"(a) p̂ \textless{} 0.40"}\NormalTok{, }\StringTok{"(b) 0.43 \textless{} p̂ \textless{} 0.57"}\NormalTok{, }\StringTok{"(c) p̂ \textgreater{} 0.54"}\NormalTok{),}
       \AttributeTok{fill =} \FunctionTok{c}\NormalTok{(}\StringTok{"lightcoral"}\NormalTok{, }\StringTok{"lightgreen"}\NormalTok{, }\StringTok{"lightblue"}\NormalTok{),}
       \AttributeTok{border =} \StringTok{"black"}\NormalTok{, }\AttributeTok{cex =} \FloatTok{0.8}\NormalTok{)}
\end{Highlighting}
\end{Shaded}

\includegraphics{FA5_Sinocruz-Part_files/figure-latex/unnamed-chunk-1-1.pdf}

\subsubsection{Interpretation}\label{interpretation}

\begin{enumerate}
\def\labelenumi{\arabic{enumi}.}
\item
  There's only about a 0.82\% chance that less than 40\% will be boys.
\item
  There's a 98\% chance that between 43\% and 57\% will be girls.
\item
  There's an 11.5\% chance that more than 54\% will be boys.
\end{enumerate}

\subsection{Problem 8.49}\label{problem-8.49}

\begin{Shaded}
\begin{Highlighting}[]
\CommentTok{\# Define population values and probabilities}
\NormalTok{x }\OtherTok{\textless{}{-}} \FunctionTok{c}\NormalTok{(}\DecValTok{6}\NormalTok{, }\DecValTok{9}\NormalTok{, }\DecValTok{12}\NormalTok{, }\DecValTok{15}\NormalTok{, }\DecValTok{18}\NormalTok{)}
\NormalTok{probab }\OtherTok{\textless{}{-}} \FunctionTok{c}\NormalTok{(}\FloatTok{0.1}\NormalTok{, }\FloatTok{0.2}\NormalTok{, }\FloatTok{0.4}\NormalTok{, }\FloatTok{0.2}\NormalTok{, }\FloatTok{0.1}\NormalTok{)}
\NormalTok{n }\OtherTok{\textless{}{-}} \DecValTok{2}

\CommentTok{\# Population mean and variance}
\NormalTok{mu }\OtherTok{\textless{}{-}} \FunctionTok{sum}\NormalTok{(x }\SpecialCharTok{*}\NormalTok{ probab)}
\NormalTok{sigma2 }\OtherTok{\textless{}{-}} \FunctionTok{sum}\NormalTok{(probab }\SpecialCharTok{*}\NormalTok{ (x }\SpecialCharTok{{-}}\NormalTok{ mu)}\SpecialCharTok{\^{}}\DecValTok{2}\NormalTok{)}

\FunctionTok{cat}\NormalTok{(}\StringTok{"Population mean (mu) ="}\NormalTok{, mu, }\StringTok{"}\SpecialCharTok{\textbackslash{}n}\StringTok{"}\NormalTok{)}
\end{Highlighting}
\end{Shaded}

\begin{verbatim}
## Population mean (mu) = 12
\end{verbatim}

\begin{Shaded}
\begin{Highlighting}[]
\FunctionTok{cat}\NormalTok{(}\StringTok{"Population variance (sigma\^{}2) ="}\NormalTok{, sigma2, }\StringTok{"}\SpecialCharTok{\textbackslash{}n\textbackslash{}n}\StringTok{"}\NormalTok{)}
\end{Highlighting}
\end{Shaded}

\begin{verbatim}
## Population variance (sigma^2) = 10.8
\end{verbatim}

\begin{Shaded}
\begin{Highlighting}[]
\NormalTok{samples }\OtherTok{\textless{}{-}} \FunctionTok{expand.grid}\NormalTok{(}\AttributeTok{x1 =}\NormalTok{ x, }\AttributeTok{x2 =}\NormalTok{ x, }\AttributeTok{KEEP.OUT.ATTRS =} \ConstantTok{FALSE}\NormalTok{)}
\NormalTok{samples}\SpecialCharTok{$}\NormalTok{mean }\OtherTok{\textless{}{-}} \FunctionTok{rowMeans}\NormalTok{(samples)}

\CommentTok{\# Lookup for probabilities}
\NormalTok{p\_lookup }\OtherTok{\textless{}{-}} \FunctionTok{setNames}\NormalTok{(probab, x)}
\NormalTok{samples}\SpecialCharTok{$}\NormalTok{prob }\OtherTok{\textless{}{-}}\NormalTok{ p\_lookup[}\FunctionTok{as.character}\NormalTok{(samples}\SpecialCharTok{$}\NormalTok{x1)] }\SpecialCharTok{*}\NormalTok{ p\_lookup[}\FunctionTok{as.character}\NormalTok{(samples}\SpecialCharTok{$}\NormalTok{x2)]}

\CommentTok{\# Print the 25 samples}
\FunctionTok{print}\NormalTok{(samples)}
\end{Highlighting}
\end{Shaded}

\begin{verbatim}
##    x1 x2 mean prob
## 1   6  6  6.0 0.01
## 2   9  6  7.5 0.02
## 3  12  6  9.0 0.04
## 4  15  6 10.5 0.02
## 5  18  6 12.0 0.01
## 6   6  9  7.5 0.02
## 7   9  9  9.0 0.04
## 8  12  9 10.5 0.08
## 9  15  9 12.0 0.04
## 10 18  9 13.5 0.02
## 11  6 12  9.0 0.04
## 12  9 12 10.5 0.08
## 13 12 12 12.0 0.16
## 14 15 12 13.5 0.08
## 15 18 12 15.0 0.04
## 16  6 15 10.5 0.02
## 17  9 15 12.0 0.04
## 18 12 15 13.5 0.08
## 19 15 15 15.0 0.04
## 20 18 15 16.5 0.02
## 21  6 18 12.0 0.01
## 22  9 18 13.5 0.02
## 23 12 18 15.0 0.04
## 24 15 18 16.5 0.02
## 25 18 18 18.0 0.01
\end{verbatim}

\begin{Shaded}
\begin{Highlighting}[]
\CommentTok{\# Sampling distribution of the sample mean (group identical means)}
\NormalTok{dist\_mean }\OtherTok{\textless{}{-}} \FunctionTok{aggregate}\NormalTok{(prob }\SpecialCharTok{\textasciitilde{}}\NormalTok{ mean, }\AttributeTok{data =}\NormalTok{ samples, }\AttributeTok{FUN =}\NormalTok{ sum)}
\NormalTok{dist\_mean }\OtherTok{\textless{}{-}}\NormalTok{ dist\_mean[}\FunctionTok{order}\NormalTok{(dist\_mean}\SpecialCharTok{$}\NormalTok{mean), ]}
\FunctionTok{print}\NormalTok{(dist\_mean)}
\end{Highlighting}
\end{Shaded}

\begin{verbatim}
##   mean prob
## 1  6.0 0.01
## 2  7.5 0.04
## 3  9.0 0.12
## 4 10.5 0.20
## 5 12.0 0.26
## 6 13.5 0.20
## 7 15.0 0.12
## 8 16.5 0.04
## 9 18.0 0.01
\end{verbatim}

\begin{Shaded}
\begin{Highlighting}[]
\CommentTok{\# Checking the mean and variance of the sampling distribution}
\NormalTok{mu\_xbar }\OtherTok{\textless{}{-}} \FunctionTok{sum}\NormalTok{(dist\_mean}\SpecialCharTok{$}\NormalTok{mean }\SpecialCharTok{*}\NormalTok{ dist\_mean}\SpecialCharTok{$}\NormalTok{prob)}
\NormalTok{sigma2\_xbar }\OtherTok{\textless{}{-}} \FunctionTok{sum}\NormalTok{(dist\_mean}\SpecialCharTok{$}\NormalTok{prob }\SpecialCharTok{*}\NormalTok{ (dist\_mean}\SpecialCharTok{$}\NormalTok{mean }\SpecialCharTok{{-}}\NormalTok{ mu\_xbar)}\SpecialCharTok{\^{}}\DecValTok{2}\NormalTok{)}

\FunctionTok{cat}\NormalTok{(}\StringTok{"}\SpecialCharTok{\textbackslash{}n}\StringTok{Mean of sampling distribution (mu\_xbar) ="}\NormalTok{, mu\_xbar, }\StringTok{"}\SpecialCharTok{\textbackslash{}n}\StringTok{"}\NormalTok{)}
\end{Highlighting}
\end{Shaded}

\begin{verbatim}
## 
## Mean of sampling distribution (mu_xbar) = 12
\end{verbatim}

\begin{Shaded}
\begin{Highlighting}[]
\FunctionTok{cat}\NormalTok{(}\StringTok{"Variance of sampling distribution (sigma\^{}2\_xbar) ="}\NormalTok{, sigma2\_xbar, }\StringTok{"}\SpecialCharTok{\textbackslash{}n}\StringTok{"}\NormalTok{)}
\end{Highlighting}
\end{Shaded}

\begin{verbatim}
## Variance of sampling distribution (sigma^2_xbar) = 5.4
\end{verbatim}

\begin{Shaded}
\begin{Highlighting}[]
\FunctionTok{cat}\NormalTok{(}\StringTok{"sigma\^{}2 / n ="}\NormalTok{, sigma2 }\SpecialCharTok{/}\NormalTok{ n, }\StringTok{"}\SpecialCharTok{\textbackslash{}n}\StringTok{"}\NormalTok{)}
\end{Highlighting}
\end{Shaded}

\begin{verbatim}
## sigma^2 / n = 5.4
\end{verbatim}

\subsubsection{Interpretation}\label{interpretation-1}

The population of student credit hours at Metropolitan Technological
College has an average of 12 hours with moderate variation. When we take
samples of size 2 (with replacement), the mean of all possible sample
means remains 12, confirming that the sample mean is an unbiased
estimator. However, the spread of these sample means is smaller
(variance = 5.4) than the original population (variance = 10.8), showing
that averaging reduces variability and makes sample means more reliable
indicators of the true population mean.

\end{document}
