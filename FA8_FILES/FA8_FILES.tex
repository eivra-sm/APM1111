% Options for packages loaded elsewhere
\PassOptionsToPackage{unicode}{hyperref}
\PassOptionsToPackage{hyphens}{url}
\documentclass[
]{article}
\usepackage{xcolor}
\usepackage[margin=1in]{geometry}
\usepackage{amsmath,amssymb}
\setcounter{secnumdepth}{-\maxdimen} % remove section numbering
\usepackage{iftex}
\ifPDFTeX
  \usepackage[T1]{fontenc}
  \usepackage[utf8]{inputenc}
  \usepackage{textcomp} % provide euro and other symbols
\else % if luatex or xetex
  \usepackage{unicode-math} % this also loads fontspec
  \defaultfontfeatures{Scale=MatchLowercase}
  \defaultfontfeatures[\rmfamily]{Ligatures=TeX,Scale=1}
\fi
\usepackage{lmodern}
\ifPDFTeX\else
  % xetex/luatex font selection
\fi
% Use upquote if available, for straight quotes in verbatim environments
\IfFileExists{upquote.sty}{\usepackage{upquote}}{}
\IfFileExists{microtype.sty}{% use microtype if available
  \usepackage[]{microtype}
  \UseMicrotypeSet[protrusion]{basicmath} % disable protrusion for tt fonts
}{}
\makeatletter
\@ifundefined{KOMAClassName}{% if non-KOMA class
  \IfFileExists{parskip.sty}{%
    \usepackage{parskip}
  }{% else
    \setlength{\parindent}{0pt}
    \setlength{\parskip}{6pt plus 2pt minus 1pt}}
}{% if KOMA class
  \KOMAoptions{parskip=half}}
\makeatother
\usepackage{graphicx}
\makeatletter
\newsavebox\pandoc@box
\newcommand*\pandocbounded[1]{% scales image to fit in text height/width
  \sbox\pandoc@box{#1}%
  \Gscale@div\@tempa{\textheight}{\dimexpr\ht\pandoc@box+\dp\pandoc@box\relax}%
  \Gscale@div\@tempb{\linewidth}{\wd\pandoc@box}%
  \ifdim\@tempb\p@<\@tempa\p@\let\@tempa\@tempb\fi% select the smaller of both
  \ifdim\@tempa\p@<\p@\scalebox{\@tempa}{\usebox\pandoc@box}%
  \else\usebox{\pandoc@box}%
  \fi%
}
% Set default figure placement to htbp
\def\fps@figure{htbp}
\makeatother
\setlength{\emergencystretch}{3em} % prevent overfull lines
\providecommand{\tightlist}{%
  \setlength{\itemsep}{0pt}\setlength{\parskip}{0pt}}
\usepackage{booktabs}
\usepackage{longtable}
\usepackage{array}
\usepackage{multirow}
\usepackage{wrapfig}
\usepackage{float}
\usepackage{colortbl}
\usepackage{pdflscape}
\usepackage{tabu}
\usepackage{threeparttable}
\usepackage{threeparttablex}
\usepackage[normalem]{ulem}
\usepackage{makecell}
\usepackage{xcolor}
\usepackage{bookmark}
\IfFileExists{xurl.sty}{\usepackage{xurl}}{} % add URL line breaks if available
\urlstyle{same}
\hypersetup{
  pdftitle={Formative Assessment 8},
  pdfauthor={Sinocruz, A \& Tagaytay, G},
  hidelinks,
  pdfcreator={LaTeX via pandoc}}

\title{Formative Assessment 8}
\author{Sinocruz, A \& Tagaytay, G}
\date{2025-11-26}

\begin{document}
\maketitle

github link:

\subsection{Assumptions}\label{assumptions}

\subsubsection{Assumption \#1: You have one dependent variable that is
measured at the continuous
level.}\label{assumption-1-you-have-one-dependent-variable-that-is-measured-at-the-continuous-level.}

\textbf{Remark.} The dependent variable is \emph{weight}, which measures
dried plant weight in grams. This dependent variable is continuous.

\subsubsection{Assumption \#2: You have one independent variable that
consists of three or more categorical, independent
groups.}\label{assumption-2-you-have-one-independent-variable-that-consists-of-three-or-more-categorical-independent-groups.}

\textbf{Remark.} The independent variable is \emph{group}, which has
three categorical levels: \textbf{ctrl}, \textbf{trt1}, and
\textbf{trt2}.

\subsubsection{Assumption \#3: You should have independence of
observations.}\label{assumption-3-you-should-have-independence-of-observations.}

\textbf{Remark.} Each plant was measured independently, and no plant
appears in more than one group. The groups are independent experimental
conditions.

\subsubsection{Assumption \#4: There should be no significant outliers
in the three groups of your independent variable in terms of the
dependent
variable.}\label{assumption-4-there-should-be-no-significant-outliers-in-the-three-groups-of-your-independent-variable-in-terms-of-the-dependent-variable.}

\pandocbounded{\includegraphics[keepaspectratio]{FA8_FILES_files/figure-latex/unnamed-chunk-1-1.pdf}}

\textbf{Remark.} There were no significant outliers in any of the three
groups, as assessed by visual inspection of the raincloud/boxplots.

\paragraph{Assumption \#5: The dependent variable should be
approximately normally distributed for each group of the independent
variable.}\label{assumption-5-the-dependent-variable-should-be-approximately-normally-distributed-for-each-group-of-the-independent-variable.}

\begin{longtable}[t]{cccc}
\caption{\label{tab:unnamed-chunk-2}Descriptive Statistics and Normality Tests by Group (Transposed)}\\
\toprule
\multicolumn{4}{c}{Weight (g)} \\
\cmidrule(l{3pt}r{3pt}){1-4}
 & ctrl & trt1 & trt2\\
\midrule
Valid & 10 & 10 & 10\\
Missing & 0 & 0 & 0\\
Mean & 5.032 & 4.661 & 5.526\\
SD & 0.583 & 0.794 & 0.443\\
Skewness & 0.231 & 0.474 & 0.485\\
\addlinespace
SE\_Skew & 0.775 & 0.775 & 0.775\\
Kurtosis & -1.117 & -1.105 & -1.160\\
SE\_Kurt & 1.549 & 1.549 & 1.549\\
Shapiro\_Wilk & 0.957 & 0.930 & 0.941\\
Shapiro\_p & 0.747 & 0.452 & 0.564\\
\bottomrule
\end{longtable}

\textbf{Remark.} Since all p-values are greater than .05, the plant
weights are approximately normally distributed for each group.

\paragraph{Assumption \#6: Homogeneity of variances (i.e., the variance
of the dependent variable is equal in each group of your independent
variable).}\label{assumption-6-homogeneity-of-variances-i.e.-the-variance-of-the-dependent-variable-is-equal-in-each-group-of-your-independent-variable.}

\begin{longtable}[t]{cccc}
\caption{\label{tab:unnamed-chunk-3}Test for Equality of Variances (Levene's Test)}\\
\toprule
df1 & df2 & Statistic (F) & p-value\\
\midrule
2 & 27 & 1.1192 & 0.3412\\
\bottomrule
\end{longtable}

\textbf{Remark.} There was homogeneity of variances, as assessed by
Levene's test (p \textgreater{} .05).

\paragraph{Computation}\label{computation}

\begin{longtable}[t]{ccccccc}
\caption{\label{tab:unnamed-chunk-4}ANOVA Table for Weight by Group}\\
\toprule
Source & DF & Sum of Squares & Mean Square & F & P & Partial Eta²\\
\midrule
group & 2 & 3.766 & 1.883 & 4.846 & 0.0159 & 0.264\\
Residuals & 27 & 10.492 & 0.389 & NA & NA & NA\\
\bottomrule
\end{longtable}

\begin{longtable}[t]{lccc}
\caption{\label{tab:unnamed-chunk-5}Descriptive Statistics of Weight by Group}\\
\toprule
group & Mean & SD & N\\
\midrule
ctrl & 5.032 & 0.583 & 10\\
trt1 & 4.661 & 0.794 & 10\\
trt2 & 5.526 & 0.443 & 10\\
\bottomrule
\end{longtable}

\begin{longtable}[t]{lcccccc}
\caption{\label{tab:unnamed-chunk-6}Post-hoc Tukey HSD Comparison of Weight (Plant Growth)}\\
\toprule
\multicolumn{7}{c}{95\% CI for Mean Difference} \\
\cmidrule(l{3pt}r{3pt}){1-7}
Comparison & Mean Diff & Lower & Upper & SE & t & p\_Tukey\\
\midrule
ctrl - trt1 & -0.371 & -1.062 & 0.32 & 0.279 & -1.331 & 0.391\\
ctrl - trt2 & 0.494 & -0.197 & 1.185 & 0.279 & 1.772 & 0.198\\
trt1 - trt2 & 0.865 & 0.174 & 1.556 & 0.279 & 3.103 & 0.012\\
\bottomrule
\end{longtable}
\newpage

\subsection{Reporting}\label{reporting}

A one-way ANOVA was conducted to determine whether plant weight differed
among the three treatment groups (\emph{ctrl}, \emph{trt1},
\emph{trt2}). The dependent variable was dried plant weight (in grams),
and the independent variable consisted of three categorical, independent
groups. There were no outliers in any group, as assessed by inspection
of the boxplots. Plant weight was approximately normally distributed for
each group, as assessed by Shapiro-Wilk tests (p \textgreater{} .05).
Homogeneity of variances was confirmed by Levene's test (p
\textgreater{} .05), indicating that the assumption of equal variances
was met. Descriptive statistics showed that mean plant weight differed
across groups: \textbf{ctrl} (M = 5.032, SD = 0.583), \textbf{trt1} (M =
4.661, SD = 0.794), and \textbf{trt2} (M = 5.526, SD = 0.443). The
one-way ANOVA revealed that mean plant weight was significantly
different between treatment groups, \emph{F}(2, 27) = 4.846, \emph{p} =
.016, partial \(\eta^2\) = .264. Tukey post hoc comparisons indicated
that the mean weight for \textbf{trt2} was significantly higher than for
\textbf{trt1} (mean difference = 0.864 g, 95\% CI {[}0.144, 1.584{]},
\emph{p} = .016). However, the comparisons between \textbf{ctrl
vs.~trt1} (mean difference = 0.371 g, 95\% CI {[}-0.348, 1.090{]},
\emph{p} = .436) and \textbf{ctrl vs.~trt2} (mean difference = -0.494 g,
95\% CI {[}-1.214, 0.226{]}, \emph{p} = .218) were not statistically
significant. In summary, only the comparison between \textbf{trt1 and
trt2} showed a statistically significant difference in plant weight,
with the \textbf{trt2} group producing heavier plants on average.

\end{document}
